%%%%%%%%%%%%%%%%%%%%%%%%%%%%%%%%%%%%%%%%%
% Journal Article
% LaTeX Template
% Version 1.3 (9/9/13)
%
% This template has been downloaded from:
% http://www.LaTeXTemplates.com
%
% Original author:
% Frits Wenneker (http://www.howtotex.com)
%
% License:
% CC BY-NC-SA 3.0 (http://creativecommons.org/licenses/by-nc-sa/3.0/)
%
%%%%%%%%%%%%%%%%%%%%%%%%%%%%%%%%%%%%%%%%%

%----------------------------------------------------------------------------------------
%	PACKAGES AND OTHER DOCUMENT CONFIGURATIONS
%----------------------------------------------------------------------------------------

\documentclass{article}

%\documentclass{aastex}  % version 5.0 or prior
%\usepackage{natbib}



\usepackage{graphicx}
\usepackage{lipsum} % Package to generate dummy text throughout this template
%\usepackage[sc]{mathpazo} % Use the Palatino font
\usepackage[T1]{fontenc} % Use 8-bit encoding that has 256 glyphs
\linespread{1.05} % Line spacing - Palatino needs more space between lines
\usepackage{microtype} % Slightly tweak font spacing for aesthetics

\usepackage[margin=1in,columnsep=20pt]{geometry} % Document margins
\usepackage{multicol} % Used for the two-column layout of the document
\usepackage[hang, small,labelfont=bf,up,textfont=it,up]{caption} % Custom captions under/above floats in tables or figures
\usepackage{booktabs} % Horizontal rules in tables
\usepackage{float} % Required for tables and figures in the multi-column environment - they need to be placed in specific locations with the [H] (e.g. \begin{table}[H])
\usepackage{hyperref} % For hyperlinks in the PDF
\usepackage{subcaption}

\usepackage{lettrine} % The lettrine is the first enlarged letter at the beginning of the text
\usepackage{paralist} % Used for the compactitem environment which makes bullet points with less space between them
\usepackage{amsmath}
\usepackage{abstract} % Allows abstract customization
\renewcommand{\abstractnamefont}{\normalfont\bfseries} % Set the "Abstract" text to bold
\renewcommand{\abstracttextfont}{\normalfont\small\itshape} % Set the abstract itself to small italic text

\usepackage{titlesec} % Allows customization of titles
%\renewcommand\thesection{\Roman{section}} % Roman numerals for the sections
%\renewcommand\thesubsection{\Roman{subsection}} % Roman numerals for subsections
%\renewcommand\thesubsubsection{\Alph{subsubsection}} % Roman numerals for subsections
\titleformat{\section}[block]{\Large\scshape}{\thesection}{1em}{} % Change the look of the section titles
\titleformat{\subsection}[block]{\large}{\thesubsection}{1em}{} % Change the look of the section titles
\titleformat{\subsubsection}[block]{}{\thesubsubsection}{1em}{} % Change the look of the section titles

\usepackage{fancyhdr} % Headers and footers
\pagestyle{fancy} % All pages have headers and footers
\fancyhead{} % Blank out the default header
\fancyfoot{} % Blank out the default footer
\fancyhead[C]{Montana State University \quad $\bullet$ \quad CSCI 466 Artificial Intelligence \quad $\bullet$ \quad Group 21} % Custom header text
\fancyfoot[RO,LE]{\thepage} % Custom footer text

\newcommand{\ve}[1]{\boldsymbol{\mathbf{#1}}}

%----------------------------------------------------------------------------------------
%	TITLE SECTION
%----------------------------------------------------------------------------------------

\title{\vspace{-15mm}\fontsize{24pt}{10pt}\selectfont\textbf{CSCI 446 Artificial Intelligence \\ Project 2 Final Report} \\[-2mm]} % Article title
\date{\today}
\author{
\large
\textsc{Roy Smart} \and \textsc{Nevin Leh} \and \textsc{Brian Marsh}\\[2mm] % Your name
}


%----------------------------------------------------------------------------------------

\begin{document}

\maketitle % Insert title

\thispagestyle{fancy} % All pages have headers and footers

%\begin{abstract}
%We present a novel way of performing MOSES data inversions using a
%\end{abstract}

%----------------------------------------------------------------------------------------
%	ARTICLE CONTENTS
%----------------------------------------------------------------------------------------

%\begin{multicols}{2} % Two-column layout throughout the main article text
\normalsize

\begin{abstract}
	
\end{abstract}
\section{Introduction}
	In a broad sense, machine learning refers to efforts to give computers the ability to learn without explicit instructions and encompasses a wide range of problems.  Classification in the realm of machine learning is a problem that can be solved using several algorithms, but the effectiveness of the algorithm depends greatly upon the specific dataset involved.  We implemented four algorithms to attempt to classify new data points into sets based upon previous information gained through “training.”  The algorithms used are k-Nearest Neighbors, Naïve Bayes, Tree Augmented Naïve Bayes (TAN), and Iterative Dichotomiser 3 (ID3).  Additionally, each algorithm is tested with five different datasets: the Wisconsin Breast Cancer Database, the Glass Identification Database, the Iris Plants Database, the Small Soybean Database, and the 1984 United States Congressional Voting Records Database.  The effectiveness of each algorithm is measured by the metrics of precision, recall, time-complexity, and convergence.  
\section{Datasets}
	\subsection{Discretization}
	\subsection{Stratified Cross-Validation}
	\subsection{Missing Values}
\section{$k$-Nearest Neighbors}
	\subsection{Training}
		\subsubsection{Constructing Probability Table}
	\subsection{Validation}
		\subsubsection{Value Distance Metric}
		\subsubsection{Determining $k$ and $p$}
\section{Naive Bayes}
	\subsection{Training}
		\subsubsection{Constructing Probability Table}
	\subsection{Validation}
		\subsubsection{Determining Class Probability Distribution}
\section{TAN}
	\subsection{Training}
		\subsubsection{Constructing Probability Table}
		\subsubsection{Constructing Augmented Tree}
	\subsection{Validation}
		\subsubsection{Determining Class Probability Distribution}
\section{ID3}
	\subsection{Training}
	\subsection{Validation}
\section{Results}
	\subsection{Algorithm Convergence}
	\subsection{Algorithm Precision}
\section{Conclusion}
	

	\pagebreak


	%\bibliographystyle{apj}
	\bibliographystyle{apalike}
	
	\bibliography{sources}
\end{document}
