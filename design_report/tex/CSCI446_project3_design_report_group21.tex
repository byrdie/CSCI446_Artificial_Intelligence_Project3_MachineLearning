%%%%%%%%%%%%%%%%%%%%%%%%%%%%%%%%%%%%%%%%%
% Journal Article
% LaTeX Template
% Version 1.3 (9/9/13)
%
% This template has been downloaded from:
% http://www.LaTeXTemplates.com
%
% Original author:
% Frits Wenneker (http://www.howtotex.com)
%
% License:
% CC BY-NC-SA 3.0 (http://creativecommons.org/licenses/by-nc-sa/3.0/)
%
%%%%%%%%%%%%%%%%%%%%%%%%%%%%%%%%%%%%%%%%%

%----------------------------------------------------------------------------------------
%	PACKAGES AND OTHER DOCUMENT CONFIGURATIONS
%----------------------------------------------------------------------------------------

\documentclass{article}

%\documentclass{aastex}  % version 5.0 or prior
%\usepackage{natbib}



\usepackage{graphicx}
\usepackage{lipsum} % Package to generate dummy text throughout this template
%\usepackage[sc]{mathpazo} % Use the Palatino font
\usepackage[T1]{fontenc} % Use 8-bit encoding that has 256 glyphs
\linespread{1.05} % Line spacing - Palatino needs more space between lines
\usepackage{microtype} % Slightly tweak font spacing for aesthetics

\usepackage[margin=1in,columnsep=20pt]{geometry} % Document margins
\usepackage{multicol} % Used for the two-column layout of the document
\usepackage[hang, small,labelfont=bf,up,textfont=it,up]{caption} % Custom captions under/above floats in tables or figures
\usepackage{booktabs} % Horizontal rules in tables
\usepackage{float} % Required for tables and figures in the multi-column environment - they need to be placed in specific locations with the [H] (e.g. \begin{table}[H])
\usepackage{hyperref} % For hyperlinks in the PDF
\usepackage{subcaption}

\usepackage{lettrine} % The lettrine is the first enlarged letter at the beginning of the text
\usepackage{paralist} % Used for the compactitem environment which makes bullet points with less space between them
\usepackage{amsmath}
\usepackage{abstract} % Allows abstract customization
\renewcommand{\abstractnamefont}{\normalfont\bfseries} % Set the "Abstract" text to bold
\renewcommand{\abstracttextfont}{\normalfont\small\itshape} % Set the abstract itself to small italic text

\usepackage{titlesec} % Allows customization of titles
%\renewcommand\thesection{\Roman{section}} % Roman numerals for the sections
%\renewcommand\thesubsection{\Roman{subsection}} % Roman numerals for subsections
%\renewcommand\thesubsubsection{\Alph{subsubsection}} % Roman numerals for subsections
\titleformat{\section}[block]{\LARGE\scshape}{\thesection}{1em}{} % Change the look of the section titles
\titleformat{\subsection}[block]{\Large\scshape}{\thesubsection}{1em}{} % Change the look of the section titles
\titleformat{\subsubsection}[block]{\large\scshape}{\thesubsubsection}{1em}{} % Change the look of the section titles

\usepackage{fancyhdr} % Headers and footers
\pagestyle{fancy} % All pages have headers and footers
\fancyhead{} % Blank out the default header
\fancyfoot{} % Blank out the default footer
\fancyhead[C]{Montana State University \quad $\bullet$ \quad CSCI 466 Artificial Intelligence \quad $\bullet$ \quad Group 21} % Custom header text
\fancyfoot[RO,LE]{\thepage} % Custom footer text

\newcommand{\ve}[1]{\boldsymbol{\mathbf{#1}}}

\title{\vspace{-15mm}\fontsize{24pt}{10pt}\selectfont\textbf{CSCI 446 Artificial Intelligence \\[2mm] Project 3 Design Report} } % Article title
\date{\today}
\author{
\large
\textsc{Roy Smart} \and \textsc{Nevin Leh} \and \textsc{Brian Marsh}\\[2mm] % Your name
}


%----------------------------------------------------------------------------------------

\begin{document}

	\maketitle % Insert title
	\thispagestyle{fancy} % All pages have headers and footers
	\normalsize

	\section{Introduction}
	Introduce Machine Learning Algorithm (MLA)!!!!
	\section{Datasets}
		\subsection{Dataset Representation}
			We will define a \textit{datum} to be a vector consisting of the class as the zeroth element and the associated attributes as the rest of the elements. Classes and attributes will be represented by an integer. Continuous data will be binned before it is inserted into each datum. The resolution of the bins will be a variable that will have to be tuned. Each dataset will be represented as a vector of datums.
		\subsection{Data Imputation}
			Imputation is the process of approximating missing values in the datasets. 
			To our knowledge there is only one dataset that has real missing values: the Wisconsin Breast Cancer Database. 
			The 1984 United States Congressional Voting Records Database appears to have missing values, but these can actually be interpreted as a stance on a particular issue. 
			Since the breast cancer database has a small proportion of missing values, it is appropriate to simply eliminate datums with missing values. 
			The authors assert that trying to train a MLA with imputed values would only create unnecessary bias in the network.
			
			However, it is a common real-world problem to attempt to classify an unknown, incomplete datum. 
			Therefore we will perform imputation on the validation datasets. 
			To approximate the missing values we will first try a hot-deck imputation, where missing attribute values of a given datum are constructed by selecting a random member of that datum's class and copying the value of the attribute. 
			If the hot-deck is unsuccessful, we will attempt to fill in the missing values using a regression model developed in \textit{Mathematica}.
		\subsection{Cross-validation}
			To partition the full datasets into test and training datasets, we will use 10-fold cross validation. This method partitions the data into ten \textit{folds} and uses one fold for testing data and the remaining nine folds for the training dataset. This process is repeated nine more times until every fold has been used as a test dataset.
	\section{Machine Learning Algorithms}
		\subsection{$k$-Nearest Neighbors}
		\subsection{Na\"ive Bayes}
		\subsection{TAN}
		\subsection{ID3}
	\section{Software Architecture}
	\section{Experiment Design}
		\subsection{Algorithm Accuracy}
			\subsubsection{Precision}
			\subsubsection{Recall}
			\subsubsection{Confusion}
		\subsection{Algorithm Time-Complexity}
		\subsection{Algorithm Convergence}

	%\bibliographystyle{apj}
	\bibliographystyle{unsrt}	
	\bibliography{sources}
\end{document}
